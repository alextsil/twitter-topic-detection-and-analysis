% This is "sig-alternate.tex" V2.1 April 2013
% This file should be compiled with V2.5 of "sig-alternate.cls" May 2012
%
% This example file demonstrates the use of the 'sig-alternate.cls'
% V2.5 LaTeX2e document class file. It is for those submitting
% articles to ACM Conference Proceedings WHO DO NOT WISH TO
% STRICTLY ADHERE TO THE SIGS (PUBS-BOARD-ENDORSED) STYLE.
% The 'sig-alternate.cls' file will produce a similar-looking,
% albeit, 'tighter' paper resulting in, invariably, fewer pages.
%
% ----------------------------------------------------------------------------------------------------------------
% This .tex file (and associated .cls V2.5) produces:
%       1) The Permission Statement
%       2) The Conference (location) Info information
%       3) The Copyright Line with ACM data
%       4) NO page numbers
%
% as against the acm_proc_article-sp.cls file which
% DOES NOT produce 1) thru' 3) above.
%
% Using 'sig-alternate.cls' you have control, however, from within
% the source .tex file, over both the CopyrightYear
% (defaulted to 200X) and the ACM Copyright Data
% (defaulted to X-XXXXX-XX-X/XX/XX).
% e.g.
% \CopyrightYear{2007} will cause 2007 to appear in the copyright line.
% \crdata{0-12345-67-8/90/12} will cause 0-12345-67-8/90/12 to appear in the copyright line.
%
% ---------------------------------------------------------------------------------------------------------------
% This .tex source is an example which *does* use
% the .bib file (from which the .bbl file % is produced).
% REMEMBER HOWEVER: After having produced the .bbl file,
% and prior to final submission, you *NEED* to 'insert'
% your .bbl file into your source .tex file so as to provide
% ONE 'self-contained' source file.
%
% ================= IF YOU HAVE QUESTIONS =======================
% Questions regarding the SIGS styles, SIGS policies and
% procedures, Conferences etc. should be sent to
% Adrienne Griscti (griscti@acm.org)
%
% Technical questions _only_ to
% Gerald Murray (murray@hq.acm.org)
% ===============================================================
%
% For tracking purposes - this is V2.0 - May 2012

\documentclass{sig-alternate-05-2015}


\begin{document}

% Copyright
%\setcopyright{acmcopyright}
%\setcopyright{acmlicensed}
%\setcopyright{rightsretained}
%\setcopyright{usgov}
%\setcopyright{usgovmixed}
%\setcopyright{cagov}
%\setcopyright{cagovmixed}


% DOI
\doi{10.475/123_4}

% ISBN
\isbn{123-4567-24-567/08/06}

%Conference
\conferenceinfo{PLDI '13}{June 16--19, 2013, Seattle, WA, USA}

\acmPrice{\$15.00}

%
% --- Author Metadata here ---
\conferenceinfo{WOODSTOCK}{'97 El Paso, Texas USA}
%\CopyrightYear{2007} % Allows default copyright year (20XX) to be over-ridden - IF NEED BE.
%\crdata{0-12345-67-8/90/01}  % Allows default copyright data (0-89791-88-6/97/05) to be over-ridden - IF NEED BE.
% --- End of Author Metadata ---

\title{Alternate {\ttlit ACM} SIG Proceedings Paper in LaTeX
Format\titlenote{(Produces the permission block, and
copyright information). For use with
SIG-ALTERNATE.CLS. Supported by ACM.}}
\subtitle{[Extended Abstract]
\titlenote{A full version of this paper is available as
\textit{Author's Guide to Preparing ACM SIG Proceedings Using
\LaTeX$2_\epsilon$\ and BibTeX} at
\texttt{www.acm.org/eaddress.htm}}}
%
% You need the command \numberofauthors to handle the 'placement
% and alignment' of the authors beneath the title.
%
% For aesthetic reasons, we recommend 'three authors at a time'
% i.e. three 'name/affiliation blocks' be placed beneath the title.
%
% NOTE: You are NOT restricted in how many 'rows' of
% "name/affiliations" may appear. We just ask that you restrict
% the number of 'columns' to three.
%
% Because of the available 'opening page real-estate'
% we ask you to refrain from putting more than six authors
% (two rows with three columns) beneath the article title.
% More than six makes the first-page appear very cluttered indeed.
%
% Use the \alignauthor commands to handle the names
% and affiliations for an 'aesthetic maximum' of six authors.
% Add names, affiliations, addresses for
% the seventh etc. author(s) as the argument for the
% \additionalauthors command.
% These 'additional authors' will be output/set for you
% without further effort on your part as the last section in
% the body of your article BEFORE References or any Appendices.

\numberofauthors{8} %  in this sample file, there are a *total*
% of EIGHT authors. SIX appear on the 'first-page' (for formatting
% reasons) and the remaining two appear in the \additionalauthors section.
%
\author{
% You can go ahead and credit any number of authors here,
% e.g. one 'row of three' or two rows (consisting of one row of three
% and a second row of one, two or three).
%
% The command \alignauthor (no curly braces needed) should
% precede each author name, affiliation/snail-mail address and
% e-mail address. Additionally, tag each line of
% affiliation/address with \affaddr, and tag the
% e-mail address with \email.
%
% 1st. author
\alignauthor
Ben Trovato\titlenote{Dr.~Trovato insisted his name be first.}\\
       \affaddr{Institute for Clarity in Documentation}\\
       \affaddr{1932 Wallamaloo Lane}\\
       \affaddr{Wallamaloo, New Zealand}\\
       \email{trovato@corporation.com}
% 2nd. author
\alignauthor
G.K.M. Tobin\titlenote{The secretary disavows
any knowledge of this author's actions.}\\
       \affaddr{Institute for Clarity in Documentation}\\
       \affaddr{P.O. Box 1212}\\
       \affaddr{Dublin, Ohio 43017-6221}\\
       \email{webmaster@marysville-ohio.com}
% 3rd. author
\alignauthor Lars Th{\o}rv{\"a}ld\titlenote{This author is the
one who did all the really hard work.}\\
       \affaddr{The Th{\o}rv{\"a}ld Group}\\
       \affaddr{1 Th{\o}rv{\"a}ld Circle}\\
       \affaddr{Hekla, Iceland}\\
       \email{larst@affiliation.org}
}

\maketitle
\begin{abstract}
todo todo todo todo todo todo todo todo todo todo todo todo todo todo todo todo todo todo todo todo todo todo todo todo todo todo todo todo todo todo todo todo todo todo todo todo todo todo todo todo todo todo todo todo todo todo todo todo todo todo todo todo todo todo todo todo todo todo todo todo todo todo todo todo todo todo todo todo todo todo todo todo todo todo todo todo todo todo todo todo todo todo todo todo todo todo todo todo todo todo todo todo todo todo todo todo todo todo todo todo todo todo todo todo todo todo todo todo todo todo todo todo todo todo todo todo todo todo todo todo todo todo todo todo todo todo todo todo todo todo todo todo todo todo todo todo todo todo todo todo todo todo todo todo todo todo todo todo todo todo todo todo todo todo todo todo todo todo todo todo todo todo todo todo todo todo todo todo todo todo todo todo todo todo todo todo todo todo todo todo todo todo todo todo todo todo todo todo todo todo todo todo todo todo todo todo todo todo todo todo (200 kai kati lekseis - gia na kseroume poso xoro pianei)
\end{abstract}


%
% The code below should be generated by the tool at
% http://dl.acm.org/ccs.cfm
% Please copy and paste the code instead of the example below. 
%
%\begin{}
%<ccs2012>
 %<concept>
  %<concept_id>10010520.10010553.10010562</concept_id>
  %<concept_desc>Computer systems organization~Embedded systems</concept_desc>
  %<concept_significance>500</concept_significance>
 %</concept>
 %<concept>
  %<concept_id>10010520.10010575.10010755</concept_id>
  %<concept_desc>Computer systems organization~Redundancy</concept_desc>
  %<concept_significance>300</concept_significance>
 %</concept>
 %<concept>
  %<concept_id>10010520.10010553.10010554</concept_id>
  %<concept_desc>Computer systems organization~Robotics</concept_desc>
  %<concept_significance>100</concept_significance>
 %</concept>
 %<concept>
  %<concept_id>10003033.10003083.10003095</concept_id>
  %<concept_desc>Networks~Network reliability</concept_desc>
  %<concept_significance>100</concept_significance>
 %</concept>
%</ccs2012>  
%\end{}

\ccsdesc[500]{Computer systems organization~Embedded systems}
\ccsdesc[300]{Computer systems organization~Redundancy}
\ccsdesc{Computer systems organization~Robotics}
\ccsdesc[100]{Networks~Network reliability}


%
% End generated code
%

%
%  Use this command to print the description
%
\printccsdesc

% We no longer use \terms command
%\terms{Theory}

\keywords{ACM proceedings; \LaTeX; text tagging}

\section{Introduction}
Nowadays, the World Wide Web has transformed from a large, static library that people only browse into a vast and dynamic information resource. Relying on this, social networks is a very popular and powerful tool for expressing opinions, broadcasting news, and simply communicating with friends. People using them for commenting on significant events in real time, with several hundred micro-blogs posted each second. 

The most popular micro-blogging service is Twitter. The popularity of Twitter stems from its availability on a number of different electronic devices (web and cell phones. There is a prevalence of a subculture in Twitter that encourages users to acquire a large friend pool, as well as send tweets on a wide variety of subjects, typically several times a day. 

Monitoring and analysing this rich and continuous flow of user-generated content can yield unprecedentedly valuable information, which would not have been available from traditional
media outlets. Tweets can be seen as a dynamic source of information enabling individuals, corporations, and government organizations to stay informed of "what is happening
now." For instance, people would be interested in getting advice, opinions, facts, or updates on news or events. Companies are increasingly using Twitter to advertise and recommend products, brands, and services; to build and maintain reputations; to analyse users' sentiment regarding their products (or those of their competitors); to respond to customers' complaints; and to improve decision making and business intelligence. Twitter has also emerged as a fast communication channel for gathering and spreading breaking news, for predicting election results, and for sharing political events and conversations. It has also become an important analytical tool for crime prediction and monitoring terrorist activities.

Twitter promotes an attractive style stating breaking news, as there is very little lag between the time that an event happens or is first reported in the news media and the time at which it is the subject of a posting on Twitter. Twitter can be characterized as an endless database, which collects millions of real-time short text messages every second. Tweets also have a mechanism by which the user can link to other objects on the web such as articles, images or videos which is typically used to link tweets to related material on the Internet. Thereafter, the first result is that the size of information is multiplied and the variety of references is bigger, as well. These messages are not only just data, but they can be manipulated efficiently. One well-timed subject of research is to use those messages for event detection. In other words, the tweets is a source of inventing which topics are more seasonable. Event detection has also instant impact on the world, through the quick transmission of the news and necessary briefing in some cases. 

With the passage of time and the effect of more and more users the topic acquire much popularity. Through this phenomenon we can form a general summarization of the event. This process is called event summarization. The massive crowd keeps close pace with the development of trending topics and provide the timely updated information. Twitter has shown its powerful ability in information delivery in many events, like the wildfires in San Diego and the earthquake in Japan. In response to searches for ongoing events, today's major search engines simply find tweets that match the query terms, and present the most recent ones. This approach has the advantage of leveraging existing query matching technologies, and for simple one-shot events such as earthquakes it works well. However, for events that have "structure" or are long-running, and where users are likely to want a summary of all occurrences so far, this approach is often unsatisfactory.

Event detection is a growing domain of research. Many different species of algorithms have been detected regarding this sector. A common approach are techniques that are based on text categorization. In addition, there are some methods that reclaim the display frequency of each term.

The computational treatment of sentiment has recently attracted a great deal of attention, in part because of its potential applications. One of the main reasons for sentiment analysis is the aforementioned increase of user-generated content on the Web which has resulted in a wealth of information that is potentially of vital importance to institutions and companies. Typically, document-based sentiment analysis processes operate at a particular level, i.e. at the word or sentence level, for extracting a document's sentiment. In machine learning, the most popular approach for sentiment analysis, the selection of appropriate features for representing a document is crucial. In sentiment identification at the word level different types of features have been introduced, which are either sentiment-based (e.g. words which express a specific sentiment), syntactic-based (e.g. part-of-speech and n-grams), or semantic-based (e.g. semantic word vector spaces which capture the meaning of each word).

Document-level polarity classification is not a special case of text categorization with sentiment -rather than topic- based categories. Hence, standard machine learning classification techniques, such as support vector machines (SVMs), can be applied to the entire documents themselves. Nevertheless, some researches presented a technique that it is easy to improve the accuracy, by integrating sentence-level subjectivity detection with document-level sentiment polarity.

As we know, in the machine learning approach, each classifier is trained using a collection of representative data. In contrast, the semantic-orientation approach does not require prior training; instead, it measures a word containing positive or negative sentiment. Each approach has its own benefits and drawbacks. For example, the machine learning approach tends to be more accurate, but the semantic-orientation approach has better generality. Recently, a new lexicon-enhanced method was accrued to generate a set of sentiment words based on a sentiment lexicon as a new feature dimension. It combines these sentiment features with content-free and content-specific features used in the existing machine-learning approach. In the evaluation stage, they showed that adding the new set of sentiment features can increase sentiment-classification performance.

The Internet and other communication technologies play a potentially disruptive role on the constraints imposed on social networks. These technologies reduce the overhead and cost for being introduced to new people regardless of geography, and help us stay in touch with those we know. Some have even gone so far as to call this "the end of geography", where the process of relationship formation becomes disentangled from distance altogether.

However, geography still plays an important role. The reason is because of the strong relationship between event detection and geographical location each user belongs. Twitter is a social networking website, which means that users need not be viewed in isolation, but instead can be viewed as part of a large network of other users, user groups, and user cliques. Moreover, users have some meta-data information, such as description, source location, friends, which means that the social network structure in Twitter can aid in finding users that are most likely to tweet about news belonging to a particular geographic location or region.

The rise of micro-blogging services spurred various applications to mine the data coming from those services. Many such applications could benefit from information about the location of users, but unfortunately location information is currently very sparse. The main problem is that less than 1 per cent of tweets are geo-tagged and information available from the location field in users' profiles is unreliable at best. The benefits of mining those data promises new personalized information services, including local news summarized from tweets of nearby Twitter users, the targeting of regional advertisements, spreading business information to local customers, and novel location-based applications (e.g., Twitter-based earthquake detection, which can be faster than through traditional official channels).

There is a great number of geoinference using social networks One direction has produced approaches that claim to accurately locate the majority of posts within tens of kilometres of their true locations. Another method predicts the location of an individual from a sparse set of located users with peformance that exceeds IP-based geolocation. On the side, there is also a technique that predicts locations of Twitter users at different granularities, such as city, state, or time zone, using the content of their tweets and their tweeting behaviour.

Getting started the first section of this survey is the presentation of some techniques that aim to event detection. The survey detects both algorithms that are based on text categorization and frequency display methods. The second chapter deals with techniques of sentiment analysis. We give more weight on techniques that use machine learning. The last chapter unfolds methods for location identification.


\section{Event Detection}
Given a series of twitter posts, the goal of event detection is to extract a particular event by analysing the text or hashtag of a tweet. The process of event detection is not a lenient task as tweets stream in huge volumes and the level of noise is kept high \cite{petrovic2010streaming}. On the other hand, the huge volume of the stream allows the use of streaming algorithms, thus making event detection an accomplishable task \cite{petrovic2010streaming}. 
\par
The conventional approach for this problem is to represent the documents as term frequency vectors \cite{petrovic2010streaming}. When a new document arrives, it is compared to all previous ones, and if its similarity to a specific document called "centroid" is below a threshold, the new document is registered as a new event \cite{petrovic2010streaming}. Unfortunately, this simple approach doesn't perform well as the dimensionality of the data expands. 
\par
Twitter's users usually tweet about their personal life, personal opinions, taste in music etc., which makes the majority of tweets unsuitable for our purpose. Although, when a noteworthy event occurs, a lot of users tweet about it. The goal of event detection is to detect these events in an automatic fashion, keeping the noise (tweets that are not events) as low as possible \cite{petrovic2010streaming}.
\subsection{Strategies for event detection}
As stated before, tweets stream in high rates and capture the pulse of the moment, therefore occupying us to process them right away \cite{Sankaranarayanan2009TwitterStandNI}. This is accomplished by using algorithms that are online in nature, meaning they can on inputs that are not known beforehand\cite{Sankaranarayanan2009TwitterStandNI}.
\par
Tweets are inherently noisy since most of them concern small user groups. In order to improve the results of event detection we first have to improve the input by extracting only useful information from noise \cite{Sankaranarayanan2009TwitterStandNI}. Even then, information that concerns us is still limited by the upper character limit set for each tweet, currently at 140 characters. Bad use of language, spelling errors and special puns also negatively affect the outcome of event detection.
\par
Twitter is an evolving platform where users come and go, and become friends and followers with each other. When designing event detection algorithms, one should keep in mind the above aspects and make the algorithms adapt well to new knowledge, otherwise watch them fail as time passes \cite{Sankaranarayanan2009TwitterStandNI}.
\par
The difference between newswire and Tweeter is that, in Tweeter, the identities of the news "reporters" in not known in advance and the news are not posted based on a schedule, but rather appear randomly among other tweets \cite{Sankaranarayanan2009TwitterStandNI}.
\subsubsection{Identifying users}
An approach to event detection is the identification of users that usually tweet about news \cite{Sankaranarayanan2009TwitterStandNI}. A technique to do this is to manually identify these users by looking for the most common set of followers among them \cite{Sankaranarayanan2009TwitterStandNI}.
Another technique to do this is to assume that a user with a high amount of followers represents an influential event source into a social community \cite{cataldi2013personalized}. As an example, we present figure \ref{fig:obama} , which features Barack Obama as the most influential node in this group \cite{cataldi2013personalized}.

\begin{figure}[h]
	\includegraphics[scale=0.7]{obama}
	\caption{Graph showing user importance on Twitter, based on the number of their followers. \cite{cataldi2013personalized}}
	\label{fig:obama}
\end{figure}

The added benefit of this method is that, not only we identify the users of interest but, we get an evaluation of a user's influence, meaning that an influential user can affect other users' tweet streams \cite{cataldi2013personalized}. This technique cascades as the follow-like relationships spread, making it comparable to the PageRank algorithm \cite{cataldi2013personalized}.
An interesting fact to support the above claims is that studies have shown that 10\% of Twitter's users are responsible for 90\% of the tweets \cite{Sankaranarayanan2009TwitterStandNI}.

\subsection{Term and Topic Extraction}
Although manual techniques exists, we will solely focus on automatic extraction techniques as they fit best on the Twitter paradigm. Filtering could be applied on all tweets, expect those posted by influential users that usually post news, leaving us with a guaranteed source of news. The goal of the extraction is to find a way to distinguish tweets that are clearly not news, and not completely eliminate the noise, which is an enormous task to take on \cite{Sankaranarayanan2009TwitterStandNI}. For the above task, \cite{Sankaranarayanan2009TwitterStandNI} used a naive Bayes classifier that was trained using tweets that were marked as news or not news.
\subsubsection{Setting a threshold for term extraction}
An approach to selecting emerging terms relies on a user-specified threshold parameter, as described by \cite{cataldi2013personalized}. It is assumed that, given two keywords that are likely to be news, they can be considered as such based on user evaluation. A threshold value called critical drop is introduced, which allows the user to decide when a term is news or not.

\begin{figure}
	\includegraphics[scale=0.78]{criticaldrop}
	\centering
	\caption{The critical drop cut-off: the maximum drop is the highest variation in the ordered list of weights
		(red mark). The average drop (between consecutive entities) is the average difference between those items
		that are ranked before the identified maximum drop point (yellow mark). The first drop which is higher than
		the computed average drop is called the critical drop (green mark). Figure also taken from \cite{cataldi2013personalized}.}
	\label{fig:criticaldrop}
\end{figure}

In order for the above technique to work, a proper drop value has to be set. This a be a daunting task for a user or a machine. The difficulty lies in the fact that it's difficult to quantify a threshold given an abstract context that is, on top of other things, unstable. An automatic way of doing this, as described by \cite{cataldi2013personalized}, is shown below: The main idea is to preserve terms with high frequencies but only consider those whose frequency is higher than the average frequencies of most discussed terms. A model is introduced that works as follows:
\begin{itemize}
	\item Terms are ranked in descending order based on their frequencies
	\item The maximum drop value between consecutive entries is calculated and the drop point is identified.
	\item The average drop between consecutive entities for all those keywords that are ranked before the identified maximum drop point is calculated.
	\item The first drop which is higher than the computed average drop is identified as the critical drop.
\end{itemize}

At the end of this procedure, the terms ranked before the critical drop are defined as news on the specified time interval.

\subsubsection{Clustering}
The goal of clustering techniques is to group news tweets such that each group contains tweets about a specific news topic. This problem is similar to clustering documents with the main difference that the identity of news topics is not known beforehand \cite{Sankaranarayanan2009TwitterStandNI}. This problem cannot be approached by static training sets, as news evolve over time in a random fashion. A clustering approach, presented by \cite{Sankaranarayanan2009TwitterStandNI}, goes as follows: Once a tweet arrives, it is added to a news cluster and remains there forever. This techniques allows for fast processing since tweets arrive at huge rates. This makes the algorithm online. Specifically, an algorithm called leader-follower clustering has been modified and used, which allowed for clustering both in content and time. Along with the clusters, a weighted list of keywords of each tweet is kept. Time is also an important aspect in this procedure: Timestamps are kept and can render a cluster inactive if its centroid is older than a specified threshold. In this case the cluster is locked and new tweets cannot be added. 
\par
Noise can greatly affect clustering techniques. Care should be taken when creating new clusters to ensure they are of good quality. A good way to ensure high quality is to only allow the creation of clusters by influential users with many followers \cite{Sankaranarayanan2009TwitterStandNI}. All tweets from other users should be able to join a cluster but not create one. Unfortunately, this approach might prevent news registration for cases where simple users report an incident first. A way to combat this situation effectively is to allow cluster creation by users, but only activate a cluster after a certain amount of tweets has been added to it, thus making the topic news-worthy \cite{Sankaranarayanan2009TwitterStandNI}.

\subsubsection{A different approach on Clustering}
As \cite{cataldi2013personalized} states, many clustering and classification strategies cannot by applies to this task due to the fact that they tend to ignore relationships among documents(tweets) related to a news event. \cite{cataldi2013personalized} proposes a metaphor where each term is seen as a living organism: A term's vitality status increases as users use it, and slowly degrades as references drop over time. The term's vitality is weighted based on the source of the tweet. As stated before, highly influential users are more likely to post high-importance news, hence allowing us to favour terms found in their tweets.

\section{Sentiment Analysis}
Sentiment analysis (SA) can be defined as a process that automates mining of attitudes, opinions, views and emotions from text, speech, tweets and database sources through Natural Language Processing (NLP)\cite{kharde2016sentiment}. Sentiment analysis involves classifying opinions in text into categories like "positive" or "negative" or "neutral". It's also referred as subjectivity analysis, opinion mining, and appraisal extraction. Twitter Sentiment Analysis (TSA) tackles the problem of analyzing the messages posted on Twitter in terms of the sentiments they express. Twitter is a novel domain
for SA and very challenging. One of the main challenges is the length limitation, according to which tweets can be up to 140 characters. In addition, the short length and the informal type of the medium have caused the emergence of textual informalities that are extensively encountered in Twitter. Thus, methods proposed for TSA should take into account these unique characteristics. The majority of TSA methods use a method from the field of machine learning, known
as classifier. Figure 3 shows the most typical TSA process.

\begin{figure}[h]
	\includegraphics[scale=0.23]{sentiment-extraction}
	\centering
	\caption{Typical process for sentiment classification.  \cite{giachanou2016like}.}
	\label{fig:sentiment-extraction}
\end{figure}

\par
In the literature, SA has been applied at three different levels: \textit{document}, \textit{sentence}, and \textit{entity levels}. SA at the document level aims to identify the sentiment polarity
expressed in the whole document. The sentence level SA aims to classify each sentence as positive or negative, whereas entity-level SA detects the sentiment polarity of a
specific entity/target of a particular object.
Due to the length limitation, the majority of tweets contains a single sentence. Therefore, for the task of TSA there is no fundamental difference between document and sentence level. In case of tweets, SA can be applied on two levels: \textit{message/sentence} and \textit{entity levels}.

\begin{figure}[h]
	\includegraphics[scale=0.70]{schema}
	\centering
	\caption{Levels of sentiment analysis related to tweets.}
	\label{fig:schema}
\end{figure}
Four different classes can be identified in the literature of TSA:

\begin{itemize}
	\item Machine Learning
	\item Lexicon-Based
	\item Hybrid (Machine Learning and Lexicon-Based)
	\item Graph-Based
\end{itemize}

\subsection{Machine Learning}
The majority of the proposed methods that deal with TSA employs a classifier from the field of machine learning that is trained on various features of tweets. Some of the most
applied classifiers are the Naive Bayes (NB), Maximum Entropy (MaxEnt), Support Vector Machines (SVM), Multinomial Naive Bayes (MNB), Logistic Regression (LR), Random Forest (RF), and Conditional Random Field (CRF).
\par
\cite{tang2014learning} proposes learning \textit{sentiment specific word embedding} (SSWE) for sentiment analysis. It develops \textit{neural} networks and maps each \textit{ngram} to the sentiment polarity of sentence. This allows it to focus on learning the meaning of word, namely word embedding, from massive distant-supervised tweets by doing this from scratch. The effectiveness of SSWE has been implicitly evaluated by using it as features in sentiment classification on the benchmark dataset in \textit{SemEval 2013}, and explicitly verified by measuring word similarity
in the embedding space for sentiment lexicons \cite{tang2014learning}. 
\par
A deep learning approach that also uses neural networks is presented in \cite{severyn2015unitn}. This approach focuses on sentiment analysis on both message and phrase levels. The resulting model of this paper demonstrates state-of-the-art performance on both levels.
\par
One of the most used classifiers for addressing TSA is the SVM classifier. [Bakliwal et al. 2012] employed an SVM  classifier trained on 11 features to address TSA. They employed different pre-processing techniques one by one in order to measure their effectiveness. \textit{Spelling correction}, \textit{stemming}, and \textit{stop-words removal} managed to increase the accuracy of the classifier. Two different datasets were used to evaluate their approach, the Stanford dataset [Go et al. 2009] and the Mejaj [Bora 2012]. The best results were achieved by the combination of NLP- and Twitter-specific features.

\begin{figure}[h]
	\includegraphics[scale=0.40]{tokenization}
	\centering
	\caption{Typical scenarios of preprocessing on standard text and tweets \cite{giachanou2016like}.}
	\label{fig:tokenization}
\end{figure}

\subsection{Lexicon-Based}
Lexicon-based methods leverage lists of words annotated by polarity or polarity score to determine the overall opinion score of a given text. The main advantage of these methods is that they do not require training data. Lexicon-based approaches have been extensively applied on conventional text such as blogs, forums, and product reviews. However, they are less explored in TSA compared to machine-learning methods. The main reason is the uniqueness of the text on Twitter that not only contains a large number of textual peculiarities and
colloquial expressions such as \textit{yolo} and \textit{gr8} but also has a dynamic nature with new expressions and hashtags emerging from time to time. 
\par
Ortega et al. [2013] proposed a three-step technique for TSA. Pre-processing was performed in the first step and polarity detection in the second step. In the last step, they performed rule-based classification. Polarity detection and rule-based classification were based on WordNet and SentiWordNet. Their approach managed to achieve good results when evaluated on the SemEval-2013 dataset [Nakov et al. 2013]. However,the authors did not compare their method with existing techniques to prove its effectiveness. \cite{dang2010lexicon} introduced some new sentiment features. The results showed that adding these newly created features, which are often used in the existing semantic-orientation approach, and the content-free and content-specific features that come from the existing machine-learning approach can improve sentiment-classification performance significantly.

\begin{figure}[h]
	\includegraphics[scale=0.60]{lexicon-token}
	\centering
	\caption{An example of the architecture of a framework of lexicon-based sentiment analysis.}
	\label{fig:lexicon-token}
\end{figure}

\par
Another approach using a newly proposed methodology, that is lexicon-based, for capturing people's emotions in real time is presented in \cite{chatzakou2013micro}. This automated technique provides the means for an automated real-time analysis. The evaluation of the experimental results proved that the proposed method managed to capture efficiently
the emotions expressed in tweets as well as their intensity.

\subsection{Hybrid}
A number of researchers combined machine-learning and lexicon-based approaches. 
\par
An interesting hybrid method was presented by Ghiassi et al. [2013], who combined dynamic artificial neural network with n-gram. Emoticons and tweets that contained the word love or hate or their synonyms were used as features to build the two classifiers: SVM and a Dynamic Architecture for Artificial Neural Networks (DAN2). The proposed approach was tested on a collection of tweets crawled using the subject \textit{Justin Bieber}. The results showed that DAN2 managed to outperform SVM.
\par
Khuc et al. [2012] also combined a lexicon-based approach with a classifier to improve TSA accuracy. They considered the MapReduce framework to create a co-occurrence
matrix based on bigram phrases. The cosine similarity between words is then computed and the edges with low cosine score are removed. Then, they combined the score generated using a simple lexicon-based approach with a classifier. For the machine learning algorithm, they used the Online LR approach. Their experiments showed that the hybrid approach outperformed the simple lexicon-based classifier that was only
based on words/phrases that indicated sentiment.
\par
Another older interesting study was presented by Zhang et al. [2011], who proposed a hybrid method to address entity-based TSA. For each of the entities Obama, Harry Potter, Tangled, iPad, and Packers they computed a sentiment score based on their proximity to words from a sentiment lexicon. They proposed a rule-based algorithm that also considered
\textit{comparative judgments}, \textit{negation}, and \textit{expressions} that were likely to change the orientation of a phrase. To collect more annotated data and enhance the recall of the proposed method, they identified additional subjective terms using \textit{Chi-square}. The SVM classifier was then applied for sentiment polarity detection.

\subsection{Graph-Based}
Although machine-learning methods achieve a great performance on TSA, they require a large number of annotated data. Label propagation is a method that can reduce the demand of the annotated data. For this reason, a number of researchers utilized the Twitter social graph under the assumption that people influence one another. Label propagation is a \textit{semi-supervised} method in which labels are distributed to nodes using the connection graphs.
\par
Some of the first to apply a label propagation method for TSA were Speriosu et al. 2011. Their proposed method leveraged the Twitter follower graph under the assumption that people influence one another. Users, tweets, unigrams, bigrams, hashtags, and emoticons were used as nodes for the construction of the graph. The proposed label propagation method outperformed a lexicon-based approach and a MaxEnt classifier.
\par
Cui et al. [2011] tackled TSA with a label propagation method based on analysis of emotion tokens. At first, the emotion tokens from tweets were extracted. A graph propagation method was then used to assign polarities to the tokens. In the final
step, they analyzed and classified the emotion tokens. The emotion tokens included \textit{emoticons}, \textit{repeating punctuations}, and \textit{repeating letters}. Their approach managed to perform well in analyzing sentiment of messages written in any natural language.
\par
Instead of using emotion tokens, Wang et al. [2011] proposed a graph-based model that leveraged co-occurrence of hashtags to classify the sentiment of certain hashtags. They proposed different algorithms that were compared to an SVM voting. The SVM was trained with several features, including unigrams, punctuation, and emoticons.
The Loopy Belief Propagation algorithm managed to achieve the best performance in terms of accuracy compared to the other tested methods.

\begin{figure}[h]
	\includegraphics[scale=0.50]{hashtag}
	\centering
	\caption{An example of a Hashtag Graph Model. [wang 2001]}
	\label{fig:hashtag}
\end{figure}

\section{Conclusions}
%\end{document}  % This is where a 'short' article might terminate

%ACKNOWLEDGMENTS are optional
\section{Acknowledgments}


\bibliographystyle{abbrv}
\bibliography{sigproc}  % sigproc.bib is the name of the Bibliography in this case
\nocite{*} %auto einai gia na bgazei to reference xwris na xreiazetai na to kanoume cite kapou.

\end{document}
